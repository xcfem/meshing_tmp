%
% Command to print 1/2 in math mode real nice
%
\newcommand{\myonehalf}{{}^1 \!\!  /  \! {}_2}

%
% Command to print over/under left aligned in math mode
%
\newcommand{\myoverunderleft}[2]{ \begin{array}{l} #1 \\ \scriptstyle #2 \end{array} }

%
% Command to number equations 1.a, 1.b etc.
%
\newcounter{saveeqn}
\newcommand{\alpheqn}{\setcounter{saveeqn}{\value{equation}}
\stepcounter{saveeqn}\setcounter{equation}{0}
\renewcommand{\theequation}
	{\mbox{\arabic{saveeqn}.\alph{equation}}}}
\newcommand{\reseteqn}{\setcounter{equation}{\value{saveeqn}}
\renewcommand{\theequation}{\arabic{equation}}}

%
% Shorthand macros for setting up a matrix or vector
%
\newcommand{\bmat}[1]{\left[ \begin{array}{#1}}
\newcommand{\emat}{\end{array} \right]}

%
% Command for a good looking \Re in math enviornment
%
\newcommand{\RE}{\mbox{\textbf{I}}\hspace{-0.6ex}\mbox{\textbf{R}}}
\newcommand{\CPLX}{\mbox{\textbf{C}}\hspace{-1.1ex}\mbox{l}\hspace{0.4ex}}

%
% Commands for Jacobians
%
\newcommand{\Jac}[2]{\displaystyle{\frac{\partial #1}{\partial #2}}}
\newcommand{\jac}[2]{\partial #1 / \partial #2}

%
% Commands for Hessians
%
\newcommand{\Hess}[2]{\displaystyle{\frac{\partial^2 #1}{\partial #2^2}}}
\newcommand{\hess}[2]{\partial^2 #1 / \partial #2^2}
\newcommand{\HessTwo}[3]{\displaystyle{\frac{\partial^2 #1}{\partial #2 \partial #3}}}
\newcommand{\hessTwo}[3]{\partial^2 #1 / (\partial #2 \partial #3)}



%\newcommand{\Hess2}[3]{\displaystyle{\frac{\partial^2 #1}{\partial #2 \partial #3}}}
%\newcommand{\myHess2}{\frac{a}{b}}
%\newcommand{\hess2}[3]{\partial^2 #1 / (\partial #2 \partial #3)}

%
% Shorthand macros for setting up a single tab indent
%
\newcommand{\bifthen}{\begin{tabbing} xxxx\=xxxx\=xxxx\=xxxx\=xxxx\=xxxx\= \kill}
\newcommand{\eifthen}{\end{tabbing}}

%
% Shorthand for inserting four spaces
%
\newcommand{\tb}{\hspace{4ex}}

%
% Commands for beginning and ending single spacing
%
\newcommand{\bsinglespace}{\renewcommand{\baselinestretch}{1.2}\small\normalsize}
\newcommand{\esinglespace}{}

%
% Command for text font
%

\newcommand{\ttt}[1]{\texttt{#1}}

\newcommand{\tit}[1]{\textit{#1}}

\newcommand{\titbf}[1]{\textit{\textbf{#1}}}
